% Created 2023-04-07 Fri 14:55
% Intended LaTeX compiler: pdflatex
\documentclass[11pt]{article}
\usepackage[utf8]{inputenc}
\usepackage[T1]{fontenc}
\usepackage{graphicx}
\usepackage{longtable}
\usepackage{wrapfig}
\usepackage{rotating}
\usepackage[normalem]{ulem}
\usepackage{amsmath}
\usepackage{amssymb}
\usepackage{capt-of}
\usepackage{hyperref}
\usepackage{fullpage}
\usepackage[avoid-all]{widows-and-orphans}
\usepackage{svg}
\author{Sean Zhang}
\date{\today}
\title{Literate Programming}
\hypersetup{
 pdfauthor={Sean Zhang},
 pdftitle={Literate Programming},
 pdfkeywords={},
 pdfsubject={},
 pdfcreator={Emacs 28.2 (Org mode 9.5.5)}, 
 pdflang={English}}
\begin{document}

\maketitle
\tableofcontents

While reading \href{https://www.pbr-book.org/3ed-2018/Introduction/Literate\_Programming}{Physically Based Rendering}, I came across this section on
literate programming.

\section{My Projects}
\label{sec:org465184b}
I found out about this after I had already finished my first ray tracing project,
and I had already started two others.

My interpretation of this is that you explain pieces of code, which after you
can explain how they come together.

And fortunately, I wrote this all in Org mode, which seems to
\href{https://orgmode.org/manual/Extracting-Source-Code.html}{support literate programming}.

\section{Testing}
\label{sec:org561ba03}
To do this, I need a code block and a few header args -
I want to ultimately export the result as a file.

So here are some imports that I may want to explain.
\begin{verbatim}
#include <iostream>
#include <vector>
\end{verbatim}


Now that I am done explaining, I'll put it in my very skeleton main function.
\begin{verbatim}
<<imports>>

int main() {
  return 0;
}
\end{verbatim}
\end{document}